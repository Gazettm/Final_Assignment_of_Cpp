\documentclass[UTF8]{ctexart}
\usepackage{listings}
\usepackage{xcolor}
\usepackage{graphicx}
\usepackage{hyperref}
\usepackage{amsmath}
\usepackage{amssymb}
\usepackage{geometry}
\geometry{a4paper, left=2cm, right=2cm, top=2.5cm, bottom=2.5cm}

\definecolor{codegreen}{rgb}{0,0.6,0}
\definecolor{codegray}{rgb}{0.5,0.5,0.5}
\definecolor{codepurple}{rgb}{0.58,0,0.82}
\definecolor{backcolour}{rgb}{0.95,0.95,0.92}

\lstdefinestyle{mystyle}{
    backgroundcolor=\color{backcolour},   
    commentstyle=\color{codegreen},
    keywordstyle=\color{magenta},
    numberstyle=\tiny\color{codegray},
    stringstyle=\color{codepurple},
    basicstyle=\ttfamily\footnotesize,
    breakatwhitespace=false,         
    breaklines=true,                 
    captionpos=b,                    
    keepspaces=true,                 
    numbers=left,                    
    numbersep=5pt,                  
    showspaces=false,                
    showstringspaces=false,
    showtabs=false,                  
    tabsize=2
}

\lstset{style=mystyle}

\title{五子棋游戏技术文档}
\author{Gazettm and pajiiii}
\date{\today}

\begin{document}
\maketitle
\tableofcontents

\section{项目概述}
本项目是一个基于Qt框架的五子棋游戏,支持以下功能:
\begin{itemize}
    \item 双人对战模式
    \item 人机对战模式(AI玩家)
    \item 胜负判定与游戏结束处理
    \item 胜率统计与历史记录
    \item 图形化棋盘界面
\end{itemize}

项目采用MVC架构设计:
\begin{itemize}
    \item \textbf{模型(Model)}: GomokuBoard类管理棋盘状态
    \item \textbf{视图(View)}: GameWindow类处理界面渲染
    \item \textbf{控制器(Controller)}: GameWindow类处理用户输入和游戏逻辑
\end{itemize}

\section{类设计与实现}
\subsection{GomokuBoard类}
棋盘核心逻辑,管理游戏状态。

\subsubsection{成员变量}
\begin{itemize}
    \item \texttt{m\_size}: 棋盘尺寸(默认15×15)
    \item \texttt{m\_board}: 二维向量存储棋子状态
\end{itemize}

\subsubsection{关键方法}
\begin{lstlisting}[language=C++, caption=gomokuboard.h]
enum Piece { Empty, Black, White };
bool placePiece(int x, int y, Piece piece); // 落子
bool checkWin(int x, int y) const;          // 胜负判定
void reset();                               // 重置棋盘
\end{lstlisting}

胜负判定算法:
\begin{lstlisting}[language=C++]
bool GomokuBoard::checkWin(int x, int y) const {
    const int directions[4][2] = {{1,0}, {0,1}, {1,1}, {1,-1}};
    for (auto &dir : directions) {
        int Count = 1;
        // 双向检测连子数量
        for (int i = 1; i < 5; ++i) { // 正向检测
            int nx = x + dir[0] * i, ny = y + dir[1] * i;
            if (nx < 0 || nx >= size() || ny < 0 || ny >= size()) break;
            if (pieceAt(nx, ny) == currentPiece) Count++;
            else break;
        }
        for (int i = 1; i < 5; ++i) { // 反向检测
            int nx = x - dir[0] * i, ny = y - dir[1] * i;
            if (nx < 0 || nx >= size() || ny < 0 || ny >= size()) break;
            if (pieceAt(nx, ny) == currentPiece) Count++;
            else break;
        }
        if(Count >= 5) return true; // 五连珠获胜
    }
    return false;
}
\end{lstlisting}

\subsection{AiPlayer类}
实现AI玩家逻辑,包含位置评估和落子决策。

\subsubsection{核心方法}
\begin{lstlisting}[language=C++, caption=aiplayer.h]
QPoint calculateAIMove(GomokuBoard m_board); // 计算AI落子位置
int evaluatePosition(int x, int y, GomokuBoard::Piece aiPiece, GomokuBoard m_board); // 位置评估
\end{lstlisting}

评估函数实现:
\begin{lstlisting}[language=C++]
int AiPlayer::evaluatePosition(int x, int y, 
    GomokuBoard::Piece aiPiece, GomokuBoard m_board) {
    
    // 防守评分(人类玩家威胁)
    if (humanCount >= 4) score += 100000;   // 阻断四连
    else if (humanCount == 3 && !blocked) score += 10000; // 阻断活三
    
    // 进攻评分(AI连珠)
    if(aiCount == 5) score += 999999;      // 五连绝杀
    else if (aiCount >= 4) score += 5000;   // 四连
    
    // 中心区域加成
    int center = m_board.size() / 2;
    int distance = std::abs(x - center) + std::abs(y - center);
    score += (m_board.size() - distance) * 10;
}
\end{lstlisting}

\subsection{GameWindow类}
主游戏窗口,处理界面和游戏流程。

\subsubsection{游戏流程控制}
\begin{lstlisting}[language=C++]
// 人机对战流程
void GameWindow::mousePressEvent(QMouseEvent *event) {
    if (m_gameMode == HumanVsAI && m_currentPiece != Black) 
        return; // AI回合忽略点击
    
    // 玩家落子
    if (m_board.placePiece(x, y, m_currentPiece)) {
        if (checkWin) ShowWinner(); // 胜负判定
        else {
            m_currentPiece = White; // 切换到AI
            QTimer::singleShot(500, [this]() { // AI延迟落子
                QPoint aiMove = m_aiplayer.calculateAIMove(m_board);
                m_board.placePiece(aiMove.x(), aiMove.y(), White);
                if (checkWin) ShowWinner();
                else m_currentPiece = Black; // 切回玩家
            });
        }
    }
}
\end{lstlisting}

\subsection{Rating类}
胜率统计系统,基于文件存储历史记录。

\subsubsection{数据结构}
\begin{itemize}
    \item \texttt{rating}: 胜率百分比
    \item \texttt{Y}: 胜利次数
    \item \texttt{N}: 失败次数
    \item \texttt{message}: 胜率显示信息
\end{itemize}

\subsubsection{文件存储格式}
\begin{lstlisting}
Y  // 人类玩家获胜记录
N  // AI获胜记录
\end{lstlisting}

\section{关键算法}
\subsection{AI评估算法}
采用基于规则的评估函数,考虑因素:
\begin{enumerate}
    \item \textbf{防守优先级}:优先阻断对手四连(100000分)
    \item \textbf{进攻机会}:构建自身连珠(优先构建五连999999分)
    \item \textbf{位置价值}:中心区域权重更高
\end{enumerate}

评分权重矩阵:
\[
\text{Score} = \sum_{\text{方向}} \begin{cases} 
100000 & \text{对手4连} \\
10000 & \text{对手活3} \\
5000 & \text{AI活3/4连} \\
500 & \text{AI活2}
\end{cases} + 10 \times (n - |x - c| - |y - c|)
\]

\subsection{胜率统计算法}
\[
\text{胜率} = \frac{Y}{Y + N} \times 100\%
\]
\begin{itemize}
    \item $Y$:从Rating.txt读取的"Y"行数
    \item $N$:从Rating.txt读取的"N"行数
\end{itemize}

\section{用户界面设计}
界面组件与功能:
\begin{itemize}
    \item \textbf{模式选择对话框}:游戏启动时选择人机/双人模式
    \item \textbf{棋盘绘制}:使用QPainter绘制15×15网格
    \item \textbf{棋子渲染}:黑色实心圆(玩家),白色空心圆(AI)
    \item \textbf{胜率对话框}:显示历史胜率并提供清空选项
    \item \textbf{获胜提示}:模态对话框显示获胜方
\end{itemize}

界面操作流程:
\begin{enumerate}
    \item 启动游戏 → 选择对战模式
    \item 显示历史胜率(可选择清空)
    \item 玩家点击落子(人机模式下AI自动响应)
    \item 五连珠时显示获胜对话框
    \item 游戏结束记录胜率
\end{enumerate}

\section{编译运行说明}
\subsection{环境要求}
\begin{itemize}
    \item Qt 5.15.15
    \item c++ (Debian 14.2.0-19) 14.2.0
    \item g++ (Debian 14.2.0-19) 14.2.0
\end{itemize}

\subsection{编译步骤}
\begin{enumerate}
	\item 写makefile文件:
\begin{lstlisting}[language=make]
make
\end{lstlisting}
    \item 编译项目(提前写的makefile文件):
\begin{lstlisting}[language=bash]
CXX = g++
CXXFLAGS = -std=c++11 -Wall -fPIC
QT_INCLUDE = -I/usr/include/x86_64-linux-gnu/qt5/ \
             -I/usr/include/x86_64-linux-gnu/qt5/QtWidgets \
             -I/usr/include/x86_64-linux-gnu/qt5/QtGui \
             -I/usr/include/x86_64-linux-gnu/qt5/QtCore
QT_LIBS = -lQt5Widgets -lQt5Gui -lQt5Core

SRC = main.cpp gamewindow.cpp gomokuboard.cpp aiplayer.cpp Rating.cpp
OBJ = $(SRC:.cpp=.o) moc_gamewindow.o  # 添加 moc 生成的目标文件
TARGET = gomoku

all: $(TARGET)

$(TARGET): $(OBJ)
	$(CXX) $(CXXFLAGS) $^ -o $@ $(QT_LIBS)

# 生成 moc 文件
moc_%.cpp: %.h
	moc $< -o $@

# 编译 moc 文件
moc_%.o: moc_%.cpp
	$(CXX) $(CXXFLAGS) $(QT_INCLUDE) -c $< -o $@

# 编译普通源文件
%.o: %.cpp
	$(CXX) $(CXXFLAGS) $(QT_INCLUDE) -c $< -o $@

clean:
	rm -f $(OBJ) $(TARGET) moc_*
\end{lstlisting}
    \item 运行程序:
\begin{lstlisting}[language=bash]
./gomoku
\end{lstlisting}
\end{enumerate}

\subsection{文件说明}
\begin{tabular}{|l|l|}
\hline
\textbf{文件} & \textbf{功能} \\
\hline
gamewindow.cpp & 主窗口和游戏流程控制 \\
gomokuboard.cpp & 棋盘状态管理和规则逻辑 \\
aiplayer.cpp & AI玩家决策算法 \\
rating.cpp & 胜率统计系统 \\
main.cpp & 程序入口点 \\
Rating.txt & 胜率数据存储 \\
\hline
\end{tabular}

\end{document}